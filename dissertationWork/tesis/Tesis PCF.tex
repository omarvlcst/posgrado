\documentclass[letterpaper, twoside,openright]{book}
\usepackage[spanish,es-nodecimaldot]{babel}
\usepackage[utf8]{inputenc}
\usepackage[T1]{fontenc}
\usepackage[left=2.5cm,right=2.5cm,top=3cm,bottom=3.5cm]{geometry}
\usepackage{amsfonts}
\usepackage[scr=boondoxo]{mathalpha}
\usepackage{lmodern}
\usepackage{graphicx}
\usepackage{subcaption}
\usepackage{mwe}
\usepackage{multicol, latexsym, amsmath, amssymb,amsthm}
%\usepackage{cite}
\usepackage{textcomp} %Paquete para algunos caracteres especiales
\usepackage{caption}
\usepackage{hyperref}
\usepackage{color}
\usepackage{xcolor} %Color expressions
\usepackage{cancel} %Cancel expressions
\usepackage{verbatim}
\usepackage{appendix}
%%%%%%%%%%%%%%%%%%%%%%%%%%%%%%%%%%%%%%%%%
\usepackage{biblabelurl}    % For urls in references
\usepackage{natbib}
%%%%%%%%%%%%%%%%%%%%%%%%%%%%%%%%%%%%%%%%%
% JOURNAL NAMES:
\newcommand{\aap}{Astron. Astrophys.}
\newcommand{\apj}{Ap. J.}
\newcommand{\apjl}{Astrophys. J. Lett.}
\newcommand{\apjs}{Astrophys. J. Suplement}
\newcommand{\ssr}{Space Science Reviews}
\newcommand{\nphysa}{Nucl. Phys. A}
\newcommand{\arnps}{Annu. Rev. Nucl. \& Part. Sci.}
\newcommand{\araa}{Annu. Rev. Astron. \& Astrophys.}
\newcommand{\physrep}{Phys. Rep.}
\newcommand{\prd}{Phys. Rev. D.}
\newcommand{\prl}{Phys. Rev. Lett.}
\newcommand{\pr}{Phys. Rev.}
\newcommand{\mnras}{Monthly Not. Royal Astron. Soc.}
\newcommand{\nat}{Nature}
%%%%%%%%%%%%%%%%%%%%%%%%%%%%%%%%%%%%%%%%%%%%%%%%%%%%%%%%%%%%%%%%
%%%%%%%%%%%%%%%%%%%%%%%%%%%%%%%%%%%%%%%%%%%%%%%%%%%%%%%%%%%%%%%%
%\newcommand{\ET}[1]{\textbf{\textcolor{violet}{#1}}}
%\newcommand{\MS}[1]{\textbf{\textcolor{teal}{#1}}}
\newcommand{\NOM}{}
\newcommand{\ET}{}
\newcommand{\MS}{}
\newcommand{\NO}{}
\newcommand{\q}{“}
\newcommand{\p}{_{\cdot}}
\newcommand{\x}{\text{x}}
\usepackage[skins,theorems]{tcolorbox}
\tcbset{highlight math style={enhanced,
  colframe=red,colback=white,arc=0pt,boxrule=1pt}}
\newtheorem{defn}{Definición}
\newtheorem{teo}{Teorema}
%%%%%%%%%%%%%%%%%%%%%%%%%%%%%%%%%%%%%%%%%%%%%%%%%%%%%%%%%%%%%%%%
%Table separation
\setlength{\arrayrulewidth}{0.2mm}
\setlength{\tabcolsep}{8pt}
\renewcommand{\arraystretch}{1.5}
%\includeonly{resdisc}

%Table coloring
\usepackage{colortbl}

%%%%%%%%%%%%%%%%%%%%%%%%%%%%%%%%%%%%%%%%%%%%%%%%%%%%%%%%%%%%%%%%
%%%%%%%%%%%%%%%%%%%%%%%%%%%%%%%%%%%%%%%%%%%%%%%%%%%%%%%%%%%%%%%%
%%%%%%%%%%%%%%%%%%%%%%%%%%%%%%%%%%%%%%%%%%%%%%%%%%%%%%%%%%%%%%%%
\begin{document}
%%%%%%%%%%%%%%%%%%%%%%%%%%%%%%%%%%%%%%%%%%%%%%%%%%%%%%%%%%%%%%%%
%%%%%%%%%%%%%%%%%%%%%%%%%%%%%%%%%%%%%%%%%%%%%%%%%%%%%%%%%%%%%%%%
%%%%%%%%%%%%%%%%%%%%%%%%%%%%%%%%%%%%%%%%%%%%%%%%%%%%%%%%%%%%%%%%
	
\thispagestyle{empty}

\begin{figure*}
\centering
\includegraphics[scale=0.15]{unam.png}
\end{figure*}
\begin{center}
    \Large{\textbf{UNIVERSIDAD NACIONAL AUTÓNOMA DE MÉXICO}} \\
    \large{POSGRADO EN CIENCIAS FÍSICAS}\\
    \large{INSTITUTO DE ASTRONOMÍA} \\[1cm] 

	DEFLEXIÓN GRAVITACIONAL DE LA LUZ EN ESTRELLAS ULTRACOMPACTAS\\[1cm]

	\large{\textbf{TESIS}  }\\

	\large{QUE PARA OPTAR POR EL GRADO DE:}\\

	\large{\textbf{MAESTRO EN CIENCIAS (FÍSICA)} }\\[1cm]

	\large{PRESENTA:}\\

	\large{\textbf{OMAR ELÍAS VELASCO CASTILLO}  }\\[1cm]
  
	\large{\textbf{TUTOR PRINCIPAL:}}\\
	\large{DANY PIERRE PAGE ROLLINET}\\
	INSTITUTO DE ASTRONOMÍA\\[2cm]
	\large{\textbf{COMITÉ TUTOR:}}\\
	\large{DR. DARÍO NÚÑEZ ZÚÑIGA (ICN)}\\
	\large{DR. MARCELO SALGADO RODRÍGUEZ (ICN)}\\[2cm]
  
    \large{CIUDAD UNIVERSITARIA, CDMX, Enero 2024}
\end{center}

\newpage

\begin{flushleft}

  \Large{
Director de tesis  }\\[0.5cm]

  \large{ \textbf{Dr. Dany Pierre Page Rollinet} }\\[0.5cm]


  \Large{
Miembros del jurado}\\[0.5cm]


  \large{ \textbf{Dr. Marcelo Salgado Rodríguez} }\\

  \large{ \textbf{Dr. Néstor Enrique Ortiz Madrigal} }\\
  
  \large{ \textbf{Dr. Roberto Allan Sussman Livosky} }\\

  \large{ \textbf{Dr. Emilio Tejeda Rodríguez} }\\

  \large{ \textbf{Dr. Tonatiuh Matos Chassin} }\\



\end{flushleft}


\thispagestyle{empty}



\newpage

\thispagestyle{empty}

\chapter*{Agradecimientos}

\thispagestyle{empty}

En múltiples ocasiones, hay poca oportunidad de tomarse el tiempo para agradecer a aquellas personas que logran motivarnos en el día a día y nos acompañan en el proceso académico. El dar gracias a tanta gente resulta siempre secundario y muchas veces es relegado únicamente al inconsciente y jamás externado, aun cuando existe posibilidad. En la redacción de este trabajo se incluye el espacio para hacerlo, así que buscaré del mejor modo completar esta tarea.\\

Debo primeramente agradecer a mis padres, Leticia y Helio, quienes me dieron la vida y quienes forjaron mi crecimiento, dándome valiosas lecciones diarias y dejando simiente en cada una de ellas. Es invaluable e incalculable lo que su instrucción y amor han logrado en mí, soy de ellos espejo y sin duda alguna me siento afortunado de haber sido hijo único de cada uno.\\

Hace años para mí la ciencia era un tema incipiente en mi cabeza al que apenas y me acercaba en educación secundaria. Tomar la decisión de estudiar la carrera de Física y finalmente decantarme por una Maestría dentro de la misma disciplina en el área de Gravitación y Astrofísica fue una elección que me llevó muchos años tomar y razonar, acompañado de gente que me inspiró a hacerlo y que tendió su mano de menor o mayor forma. A ellos les debo todo.\\

Debo empezar primero agradeciendo a mis maestros que más me inspiraron en el área de las ciencias exactas en el bachillerato, el Ing. Roberto Macías y el Ing. Eduardo Hugues. También al M. en I. (y deidad) Eduardo de la Vega y al Lic. Marco Ronzón, quienes también pusieron una semilla en mi formación personal en esa etapa. Posteriormente, ya en la licenciatura, las raíces empezaron a crecer de la mano del Dr. Darío Núñez, de quien es un gusto siempre permanecer cerca: toda oportunidad de compartir espacio con él es entretenida y enriquecedora. El acercamiento y celo al área de Astrofísica vino impulsado por la Dra. Julieta Fierro y el Dr. Antonio Ramírez Q.E.P.D., quienes con sus lecciones me hicieron ver el papel que juega la Física no solamente en la Tierra, sino en sus alrededores y en confines más o menos lejanos del Universo.\\

Aurora y Emmanuel fueron mis compañeros más cercanos de la Licenciatura en la Facultad. Que les quede claro, chicos, que hoy no estaría en este lugar de no haber estado acompañado de ustedes. Todos los contratiempos que tuvimos, todas esas jornadas de trabajo que pensábamos que estaba de más y nuestra redistribución del mismo, que en muchas ocasiones se nos salía de las manos controlar, cobran sentido ahora. A ustedes me debo hoy y por eso estoy muy agradecido.\\

También debo agradecer a todos los Enci-amigos, llevo una parte de ustedes conmigo en donde sea. En concreto Rigo, Aldo y Paco son personas a las que agradezco ver crecer, así como también me pone feliz que sigamos manteniendo el cariño y respeto de cuando empezamos a ser amigos desde niños.\\

De personas antes y del Posgrado, aparte de todos los anteriores, también agradezco a Carlos Reyes, a Antonio Galván, a Ulises y a Rocío Lucero quienes siempre tuvieron no solamente el tiempo, sino la atención y la actitud de acompañarme y apoyarme en cuanto fuera posible. También agradezco a mi mejor amigo en la Ciudad de México, Javier, y a mis otros amigos de la Facultad de Ciencias y la Universidad Nacional, la cual siempre ha sido mi mayor núcleo de actividad académica.\\

Ya para casi culminar, mi profundo agradecimiento va para el Dr. Dany Page, quien con su personalidad diligente, sus inteligentes retroalimentaciones e incontables lecciones; me hace asegurar que el solo hecho de pertenecer a su grupo de alumnos es un privilegio, puesto que únicamente con su ejemplo basta para poner manos a la obra. También quiero agradecer profundamente a los doctores Marcelo Salgado, Néstor Ortiz y Emilio Tejeda, por tan valiosas aportaciones en mi proceso formativo y por tomarse el tiempo para leer y mandar correcciones muy atinadas y enriquecedoras para este trabajo. Y por último, quiero agradecer también a la unidad del Instituto de Ciencias Físicas en la ciudad de Cuernavaca, quienes aunque solo me vieron partir de este viaje, ellos me ayudaron a tomar vuelo para emprender esta aventura.\\

Esta tesis y mi trabajo de Maestría fue financiada con estímulo de Beca Nacional para Maestría del programa PNPC del CONACYT.

\thispagestyle{empty}

\newpage

\thispagestyle{empty}

\newpage

\section*{Unidades, Notación y Convenciones Utilizadas}

Durante la extensión de este trabajo, se empleará a menos que sea indicado lo contrario, el sistema gaussiano de unidades físicas como es habitual en Astrofísica.\\

Para cantidades relativistas, se toman los valores para las constantes $c$ y $G$ como vienen indicados en cgs. En su defecto y siempre que sea indicado, en las teorías de Relatividad debe hacerse la conversión a unidades geométricas en las cuales $c=G=1$. Para sortear este hecho, se ha decidido emplear múltiplos del radio de Schwarzschild $R_s=2GM/c^2$.\\

Los ángulos serán explícitamente expresados en grados sexagesimales a menos que vengan evaluados o convertidos de cantidades adimensionales, a los que les corresponderán los radianes.\\

La convención utilizada normalmente en la literatura para la notación de índices en 4-vectores $\vec{\mathbf{V}}$, expresados en una base coordenada estándar adecuada, es con las letras del 
alfabeto griego y las 4 componentes $V^\alpha$, así como los 4-vectores base $\vec{\mathbf{e}}_{\alpha}$, corren desde el 0 hasta el 3: $0, 1, 2, 3$.\\

La base coordenada estándar $\vec{\mathbf{e}}_{\alpha}$ que no es necesariamente ortonormal, a menos que se le señale como tal y en cuyo caso se escribirá $\hat{\mathbf{e}}_{\alpha}$; contempla así para un 4-vector $\vec{\mathbf{V}}$ las 4 componentes $V^0, V^1, V^2\text{ y }V^3$. Además, la asignación de los índices puede ser más específica en el texto según la elección del sistema coordenado. Por ejemplo, para el caso cartesiano, se pueden denotar en vez de números del
0 al 3 las letras de las coordenadas correspondientes a cada componente: $t, x, y\text{ y }z$. O bien, para el caso de coordenadas esféricas: $t, r, \theta\text{ y }\phi$. Con lo cual, donde
se haga mención de la componente $u^t$ en su base coordenada, se estaría refiriendo de manera única a la primera componente $u^0$ y así de ahí en adelante con el resto de índices. Estas reglas son también aplicables para índices en vectores duales $u_{\alpha}$ y tensores mixtos, así como también para cualquier otra elección de sistema coordenado.\\

Nuestra convención de signos para la métrica empleada en las teorías de Relatividad será la $(-,+,+,+)$. 


\section*{Notación y simbología}

Se incluye una tabla con la notación y los símbolos que serán empleadas a lo largo de este trabajo en la página siguiente, solamente se eligieron para este fin los más representativos y que permitan seguir fielmente la lectura a lo largo de todos los capítulos. \\


\begin{table}[h]
	\centering
	\begin{tabular}{lcc}
		\hline 
		& &  \\
		\textbf{Cantidad} & \textbf{Símbolo}& \\ && \\
		Velocidad de la luz & $c$ & \\&& \\
		Constante de la gravitación universal & $G$ & \\&& \\
		Masa gravitacional de un \MS{objeto compacto astrofísico (estrella o agujero negro)} & $M$ &\\&& \\
		Masa Solar & $M_{\odot}$ &\\&& \\
		Radio de una estrella & $R^*$ & \\ &&\\
		Radio de Schwarzschild de una estrella & $R_{s}$ &\\&& \\
		Presión, densidad de energía (de un fluido de materia de estrella de neutrones) & $p,\varepsilon$ & \\&&\\
		Curva situada en el espacio-tiempo & $\mathcal{C}$ &\\&& \\
		Cono de luz de un evento centrado en el punto $p$ & $C_p$ &\\&& \\
		Corrimiento al rojo medido por un observador al infinito ($1+z$) & $e^{-\Phi(R^*)}$ \\ && \\
		Función Lagrangiana de una partícula en un sistema & $\mathcal{L}$ &\\&& \\
		Función Hamiltoniana de una partícula en un sistema & $\mathcal{H}$ &\\&& \\
		Parámetro de impacto de una trayectoria de luz saliendo de una estrella & $b$ & \\&& \\
		Radio aparente al infinito (parámetro de impacto máximo medido) & $R_\infty$ & \\&& \\
		Ángulo de salida de un fotón en su punto de emisión respecto a la normal en la superficie de la estrella & $\delta$ & \\&& \\
		Proyección perpendicular del vector de salida de un fotón en su punto de emisión & $\x$ & \\&& \\
		Ángulo que forma la magnetización de un campo $\vec{B}$ respecto a la normal en la superficie de la estrella & $\gamma$ & \\&& \\
		Número de fotones que escapan de la estrella con un campo $\vec{B}$ dipolar por unidad de área y de tiempo & $\mathcal{N}_\text{dip}$ &\\ &&\\
		Intensidad específica luminosa (monocromática) & $I_\nu$ & \\&& \\
		Densidad monocromática de flujo luminoso & $F_\nu$ & \\&& \\
		Flujo luminoso & $F$ & \\&& \\
		Luminosidad & $L$ & \\&& \\
		& & \\ 
		\hline
	\end{tabular}
	%\caption{Planck Conversions}
	\label{Table:conversions}
\end{table}





\thispagestyle{empty}



\tableofcontents

\mainmatter



\thispagestyle{empty}


%Dedicatoria
\chapter*{}
%\pagenumbering{Roman} % para comenzar la numeracion de paginas en numeros romanos
\begin{flushright}
\textit{Dedicado a mi mamá\\
Leticia Castillo, este trabajo de Maestría, \\
a quien en todo momento ha sido mi principal maestra.}
\end{flushright}

%Capítulo 1: Motivación
\include{motivacion}
\label{chap:1} 

%Capítulo 2: Un Breve Repaso de Relatividad Especial y General
\include{conceptos}
\label{chap:2} 

%Capítulo 3: Antecedentes y conceptos preliminares
\include{antecedentes}
\label{chap:3} 

%Capítulo 4: Relación masa-radio y estrellas ultracompactas
\include{ultra-new}
\label{chap:4}

%Capítulo 5: Soluciones en Relatividad General para estrellas
\include{rg}
\label{chap:5}


%Capítulo 6: Generación de las imágenes y curvas de luz
\include{luz}
\label{chap:6} 

%Capítulo 7: Resultados y Discusión
\include{resdisc}
\label{chap:7}

%\chapter{Conclusiones}
\include{conclude} 
\label{chap:8} 

%ANEXO A
\appendix
\include{anexo1}
\label{chap:anexo1} 

%ANEXO B
\include{anexo1b}
\label{chap:anexo1b} 

%ANEXO C
\include{anexo2}
\label{chap:anexo2} 

%\backmatter
\thispagestyle{empty}

%%%%
% PARA COMPILAR EL BIBTEX CON LA BIBLIOGRAFIA IMPORTADA DESDE master.bib:
% 1. Primero se debe editar el master.bib hasta que se encuentre listo, puede editarse directamente desde texstudio, kile, etc.
% 2. Guardar el archivo master.bib a su última versión.
% 3. En el archivo master.bib abierto, ir a Herramientas -> Ordenes -> BibTex, y dar clic. El proceso debe terminar con éxito.
% 4. Guardar el archivo main.tex a su última versión.
% 5. En el archivo main.tex abierto (en este caso, este mismo), repetir el paso 3 (ejecutar la orden BibTex para main.tex). El proceso debe terminar con éxito.
% 6. Nuevamente en el archivo main.tex abierto, ir a Herramientas -> Ordenes -> PDFLatex y dar clic. El proceso debe terminar con éxito.
%%%%
\addcontentsline{toc}{chapter}{Bibliografía}

\bibliographystyle{mn2e-etal-ie-in-italics}
%\bibliographystyle{aasjournal}
\bibliography{Tesis}
\nocite{*}

\thispagestyle{empty}

\end{document}