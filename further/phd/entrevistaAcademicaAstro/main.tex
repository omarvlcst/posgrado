\documentclass[12pt, letterpaper]{elegantbook}
\usepackage[spanish, es-nodecimaldot]{babel}
\usepackage{amsmath}
\usepackage{amssymb}
\usepackage{amsfonts}

\newcommand{\aap}{Astron. Astrophys.}
\newcommand{\apj}{Ap. J.}
\newcommand{\apjl}{Astrophys. J. Lett.}
\newcommand{\apjs}{Astrophys. J. Suplement}
\newcommand{\ssr}{Space Science Reviews}
\newcommand{\nphysa}{Nucl. Phys. A}
\newcommand{\arnps}{Annu. Rev. Nucl. \& Part. Sci.}
\newcommand{\araa}{Annu. Rev. Astron. \& Astrophys.}
\newcommand{\physrep}{Phys. Rep.}
\newcommand{\prd}{Phys. Rev. D.}
\newcommand{\prl}{Phys. Rev. Lett.}
\newcommand{\pr}{Phys. Rev.}
\newcommand{\mnras}{Monthly Not. Royal Astron. Soc.}
\newcommand{\nat}{Nature}

\begin{document}
\mainmatter


\chapter*{Propuesta de Tema para Entrevista Académica \\ \textit{``Estrellas ultracompactas en el contexto de Relatividad General''}}


\subsection*{Datos del Alumno Aspirante}

\begin{itemize}
\item \textbf{Nombre del alumno:} Omar Elías Velasco Castillo
\item \textbf{Número de cuenta UNAM:} 413074760
\item \textbf{Entidad académica:} Instituto de Astronomía, Ciudad Universitaria
\end{itemize}
%%%%%%%%%%%%
\subsection*{}
%%%%%%%%%%%%
El estudio de los objetos compactos estelares en Astrofísica, como lo son las enanas blancas y las estrellas de
neutrones, suele colocar a las ecuaciones de estado válidas y demás cantidades físicas relacionadas a ellas (como masas, radios estelares, momentos de inercia, densidades nucleares, periodos de rotación, etc.) únicamente en los rangos de aquellas configuraciones en donde se encuentran tales estrellas detectadas hasta la actualidad. En el caso de las estrellas de neutrones, esto comprende aproximadamente a los radios de la estrella $R^*$
de 10 a 14 km y el rango de sus masas entre 1.4 a 2.1 $M_{\odot}$, ver \textit{e.g.}, \cite{Suleiman:2021hre}.\\

Desde un trabajo publicado por \cite{1975ApJ...199..471H}, se comenzó a plantear de manera plausible la existencia de estrellas de neutrones con ecuaciones de estado suficientemente \textit{blandas} para presentar $R_s<R^*\leq 1.5R_s$, es decir, tener su radio propio $R^*$ adentro de su órbita de fotones en $R_c=1.5R_s$. Este último es el radio para la esfera de fotones que se descubre en la solución de Schwarzschild: la única órbita circular para partículas de tipo luz. \cite{1985CQGra...2..219I} denominaron por primera vez a los objetos que pudieran exhibir su propia esfera de fotones por afuera del radio del mismo objeto como \textit{ultracompactos}.\\

En la actualidad hay un número considerable de trabajos que han realizado menciones al tema y que han usado este nombre para dicha clase de objetos (ver una mención desde, \textit{e.g.} \cite{1990ApJ...362..251B}). Además su popularidad ha venido creciendo en los últimos años. Sin embargo, ha habido muy escasa preocupación en estudiar las emisiones y el comportamiento de la luz que proviene de ellos. Como un ejemplo singular, en \cite{1993ApJ...406..590N} se realizaron correcciones
relativistas al límite de Eddington para estudiar las luminosidades en destellos de rayos X que pudieran venir
de estos objetos.\\

Al día de hoy, ya se conoce una ecuación de estado para la configuración de una estrella en el límite causal $c_s/c\to 1$ para la velocidad de fase de sus ondas sonoras en su interior aproximándose a la velocidad de la luz, dentro de una distribución con simetría esférica. Este modelo admite compacidad maximal para una estrella de neutrones y muestra que puede estar en el régimen de estrella ultracompacta al alcanzar, como mínimo, un radio aproximadamente igual a $R_{M_\text{máx}}= 1.412R_s$ para su masa máxima, ver \textit{e.g.}, \cite{2012ARNPS..62..485L}. Además, hay otros cocientes de $c_s/c$ que pueden en principio admitir a estrellas ultracompactas, según se discute en \cite{PhysRevC.103.045808}; quienes muestran el trabajo más reciente en este tipo de ecuaciones de estado.\\

El trabajo aquí propuesto presenta un estudio de cómo se vería la emisión de fotones directo de la superficie de
estrellas ultracompactas, así como también una descripción de su imagen y demás pormenores astrofísicos relacionados, como el flujo bolométrico, la luminosidad, el tamaño del diámetro angular y las curvas de luz. Se encuentra una corrección relativista adecuada en estrellas ultracompactas para el flujo bolométrico y la luminosidad térmica que ya incluye el fenómeno de captura de fotones emitidos que no pueden rebasar el radio crítico $R_c=1.5R_s$ y que además devuelve ausencia de emisión cuando se tiene un agujero negro esférico de Schwarzschild al radio $R_s$. Se determina el valor teórico para el radio $R_\infty$ corrido al rojo medido hacia el infinito dentro del intervalo de 11 a 14 km (\textit{i.e.}, de 6 a 8 km aproximadamente para $R^*$) y se concluye como el que debería medirse desde la Tierra para asegurar que se tiene una estrella ultracompacta. También se encuentra el rango de luminosidades al infinito de $10^{-4}\,L_\odot$ a $1\,L_\odot$ como un rango fiable para acompañar el valor anterior. Estos valores se reportan como útiles en el futuro en caso de que se puedan determinar luminosidades en este rango y con barras de error mínimas, acompañadas de un espectro térmico con una contraparte en rayos X que nos haga determinar los valores mencionados en $R_\infty$, algo todavía imposible o no visto en la actualidad en las mediciones de binarias de rayos X. Para el modelo de curvas de luz, se considera emisión térmica con una atmósfera con campo magnético dipolar para un púlsar como describen los trabajos de \cite{1983ApJ...271..283G} y \cite{1995ApJ...442..273P}, modificando ligeramente su modelo a cuando éste ya considere una magnetización en un caso general y que no solamente apunte en la dirección de un eje de simetría respecto a la superficie de la estrella. Se obtienen gráficas de las curvas de luz para diferentes casos, en los que se sitúan distintos valores para el cambio de la temperatura en los polos y el ecuador de la estrella, el ángulo en el que apunta la magnetización y también el ángulo de visión de la superficie de la estrella respecto a nosotros en la Tierra. Se muestran finalmente imágenes de la superficie del objeto que exhiben diferentes particularidades debidas al fenómeno relativista de la deflexión gravitacional de la luz que proviene de tales estrellas. 


% template to insert figure
%
%\begin{figure}[htbp]
%  \centering
%  \includegraphics[width=0.6\textwidth]
%{figurefilenamehere}
%  \caption{caption}
%\end{figure}



\printbibliography[heading=bibintoc]


\end{document}

